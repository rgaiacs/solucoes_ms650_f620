% Filename: cap01@notas_de_aula.tex
% This code is part of 'Notas de aula n\~{a}o oficiais de MS650 e F620'
% 
% Description: This file correspond to part of the textbook using in the course.
% 
% Created: 02.08.12 10:36:23 PM
% Last Change: 02.08.12 10:36:23 PM
% 
% Authors:
% - Raniere Silva, r.gaia.cs@gmail.com
% 
% Copyright (c) 2012, Raniere Silva. All rights reserved.
% 
% This work is licensed under the Creative Commons Attribution-NonCommercial-NoDerivs 3.0 Unported License. To view a copy of this license, visit http://creativecommons.org/licenses/by-nc-nd/3.0/.
%
% This work is distributed in the hope that it will be useful, but WITHOUT ANY WARRANTY; without even the implied warranty of MERCHANTABILITY or FITNESS FOR A PARTICULAR PURPOSE.
%
\chapter{S\'{e}ries de Fourier}
A s\'{e}rie
\begin{align*}
    \frac{A_0}{2} + \sum_{n = 1}^\infty \left( A_n \cos\left( n x \right) + B_n \sin\left( n x \right) \right)
\end{align*}
\'{e} chamada uma s\'{e}rie trigonom\'{e}trica.

\begin{defi}
    A s\'{e}rie trigonom\'{e}trica
    \begin{align*}
        \frac{a_0}{2} + \sum_{n = 1}^\infty \left( a_n \cos\left( n x \right) + b_n \sin\left( n x \right) \right)
    \end{align*}
    \'{e} a s\'{e}rie de Fourier da fun\c{c}\~{a}o $f(x)$ se os coeficientes forem dados por
    \begin{align*}
        a_n &= \frac{1}{n} \int_{-\pi}^\pi f(x) \cos\left( n x \right) \id{x} && n = 0, 1, 2, \ldots \\
        b_n &= \frac{1}{\pi} \int_{-\pi}^\pi f(x) \sin\left( n x \right) \id{x} && n = 1, 2, \ldots
    \end{align*}
\end{defi}

Algumas quest\~{o}es:
\begin{enumerate}
    \item a s\'{e}rie converge?
    \item qual o intervalo de converg\^{e}ncia?
    \item qual o tipo de converg\^{e}ncia?
        \begin{exem}
            Para a converg\^{e}ncia uniforme temos que $\forall \epsilon > 0, \exists n > 0$ tal que $| \delta_m(x) - \delta_n(x)| < \epsilon$ sempre que $m, n > N, \forall x \in I$ (onde $\delta_n(x) = \sum_{k = 1}^\nu \mu_k(x)$).
        \end{exem}
    \item quais fun\c{c}\~{o}es $f(x)$ s\~{a}o respresent\'{a}veis por essa s\'{e}rie? Condi\c{c}\~{o}es sobre $f(x)$?
\end{enumerate}

Um fato not\'{a}vel \'{e} que as condi\c{c}\~{o}es para representar uma fun\c{c}\~{a}o por uma s\'{e}rie de Fourier s\~{a}o menos restritivas do que para s\'{e}ries de pot\^{e}ncias.

\begin{obs}
    Temos que $\left\{ \cos\left( n x \right), \sin\left( n x \right) \right\}$ s\~{a}o fun\c{c}\~{o}es peri\'{o}dicas com $T = 2\pi$ e portanto $f(x)$ \'{e} uma fun\c{c}\~{a}o peri\'{o}dica, i.e., $f(x + 2\pi) = f(x)$.
\end{obs}

\begin{exem}
    Para $f(x) = x^2$ temos que
    \begin{align*}
        a_0 &= \frac{1}{\pi} \int_{-\pi}^\pi x^2 \id{x} = \frac{1}{\pi} \left. \frac{x^3}{3} \right|_{-\pi}^\pi = \frac{2 \pi^2}{3}, \\
        a_n &= \frac{1}{n} \int_{\pi}^\pi x^2 \cos\left( n x \right) \id{x} \\
        &= \frac{1}{\pi} \left[ \underbrace{\left. \frac{x^2 \sin\left( n x \right)}{n} \right|_{-\pi}^\pi}_{= 0} - \frac{1}{n} \int_{-\pi}^\pi 2 x \sin\left( n x \right) \id{x} \right] \\
        &= \frac{-1}{n \pi} \left[ \left. \frac{-2 x \cos\left( n x \right)}{n} \right|_{-\pi}^\pi + \frac{1}{n} \int_{-\pi}^\pi 2 \cos\left( n x \right) \id{x} \right] \\
        &= \frac{-1}{n \pi} \left[ \frac{-2 \pi \cos\left( n \pi) \right)}{n} + \frac{2 (-\pi) \cos(n) (-\pi)}{n} + \frac{2}{n} \underbrace{\left. \frac{\sin\left( n x \right)}{n} \right|_{-\pi}^\pi}_{= 0} \right] \\
        &= \frac{-1}{n \pi} \left[ \frac{- 4 \pi}{n} \cos\left( n \pi \right) \right] \\
        &= \frac{4}{n^2} (-1)^n && n > 0 \\
        b_n &= \frac{1}{n} \int_{-\pi}^\pi x^2 \sin\left( n x \right) \id{x} \\
        &= \frac{1}{\pi} \left[ \left. \frac{- x^2 \cos\left( n x \right)}{n} \right|_{-\pi}^\pi + \frac{1}{n} \int_{-\pi}^\pi 2 x \cos\left( n x \right) \id{x} \right] \\
        &= \frac{1}{n} \left[ \frac{-\pi^2 \cos\left( n \pi \right)}{n} + \frac{\pi \cos\left( n (-\pi) \right)}{n} + \frac{1}{n} \int_{-\pi}^\pi 2 x \cos\left( n x \right) \id{x} \right] \\ 
        &= \frac{1}{n} \left[ \frac{1}{n} \left[ \underbrace{\left. \frac{2 x \sin\left( n x \right)}{n} \right|_{-\pi}^\pi}_{= 0} - \frac{1}{n} \underbrace{\int_{-\pi}^\pi 2 \sin\left( n x \right) \id{x}}_{= 0} \right] \right] \\
        &= 0
    \end{align*}

    Logo,
    \begin{align*}
        x^2 = \frac{\pi^2}{3} + \sum_{n = 1}^\infty \frac{4 (-1)^n}{n^2} \cos\left( n x \right).
    \end{align*}
\end{exem}
% TODO Terminar de incluir arquivo M2S12-1.pdf. Interrompido na p\'{a}gina 3.

% TODO Incluir arquivo M2S12-2.pdf
% TODO Incluir arquivo M2S12-3.pdf
% TODO Incluir arquivo M2S12-4.pdf
% TODO Incluir arquivo M2S12-5.pdf
% TODO Incluir arquivo M2S12-6.pdf
% TODO Incluir arquivo M2S12-7.pdf
