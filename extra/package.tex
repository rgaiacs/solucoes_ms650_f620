% 'Notas de aula não oficiais de MS650 e F620' (c) 2012, 2013 de Raniere Silva
% <ra092767@ime.unicamp.br>
%
% Este trabalho é baseado nos manuscritos das notas de aula do Professor Doutor
% Jayme Vaz Júnior. para as disciplinas MS650, Métodos de Matemática Aplicada
% II, e F620, Métodos Matemáticos da Física II, disponibilizadas em
% http://www.ime.unicamp.br/~vaz/metodos2S12.htm. É permitido a este fazer uso
% deste trabalho para qualquer fim e sem nenhuma restrição.
%
% É permitido fazer uso das criações do espírito presentes neste trabalho
% diretamente relacionadas com os manuscritos das notas de aula do Professor
% Doutor Jayme Vaz Júnior única e exclusivamente para fins educacionais.
%
% Salvo indicação em contrário, este trabalho foi licenciado com a Creative
% Commons Atribuição-CompartilhaIgual 3.0 Não Adaptada. Para ver uma cópia desta
% licença, visite http://creativecommons.org/licenses/by-sa/3.0/.
%
% Este trabalho encontra-se disponível em
% https://github.com/r-gaia-cs/solucoes_ms650_f620.
%
% Este trabalho é distribuído na esperança que possa ser útil, mas SEM NENHUMA
% GARANTIA; sem uma garantia implícita de ADEQUAÇÃO a qualquer MERCADO ou
% APLICAÇÃO EM PARTICULAR.

% Lista de pacotes utilizados.

\usepackage[utf8]{inputenc}
\usepackage[T1]{fontenc}
\usepackage[brazil]{babel}
\usepackage{indentfirst}

% Text
\usepackage{enumerate}
\usepackage{latexsym}
\usepackage{parcolumns}
\usepackage{url}
\usepackage{hyperref}
\usepackage{breakurl}
\usepackage[official]{eurosym}

% Tables
\usepackage{multicol}
\usepackage{multirow}
\usepackage{array}

% Math
\usepackage{amsmath}
\usepackage{amsthm}
\usepackage{amsfonts}
\usepackage{amssymb}
\usepackage{breqn}
\allowdisplaybreaks[4]

\newtheorem{defi}{Definição}
\newtheorem{prop}{Proposição}
\newtheorem{lem}{Lema}
\newtheorem{teo}{Teorema}
\newtheorem{exem}{Exemplo}
\newtheorem{obs}{Observação}

% Deprecated
\newcommand{\id}[1]{\, \mathrm{d}#1}
% Variable of Integration
\newcommand{\vi}[1]{\, \mathrm{d}#1}
% Deprecated
\newcommand{\devp}[2]{\frac{\partial #1}{\partial #2}}
\newcommand{\fder}[2]{\frac{\mathrm{d} #1}{\mathrm{d} #2}}
% Partial DERivative
\newcommand{\pder}[2]{\frac{\partial #1}{\partial #2}}
% Residue (complex analysis)
\DeclareMathOperator{\Res}{Res}

% Index
\usepackage{makeidx}
\makeindex

% Figures
\usepackage{pb-diagram}
\usepackage{graphicx, color}
\usepackage{subfig}
\usepackage{tikz}
\usetikzlibrary{arrows,positioning,fit,petri}
\usetikzlibrary{patterns}
\usepackage{epsfig}
