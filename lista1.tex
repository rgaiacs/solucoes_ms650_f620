% Filename: lista1.tex
% 
% This code is part of 'Solutions for MS650, M\'{e}todos de Matem\'{a}tica Aplicada II, and F620, M\'{e}todos Matem\'{a}ticos da F\'{i}sica II'
% 
% Description: This file corresponds to the solutions of homework sheet 1.
% 
% Created: 14.07.12 11:13:03 AM
% Last Change: 19.07.12 08:36:46 AM
% 
% Authors:
% - Raniere Silva (2012): initial version
% 
% Copyright (c) 2012 Raniere Silva <r.gaia.cs@gmail.com>
% 
% This work is licensed under the Creative Commons Attribution-ShareAlike 3.0 Unported License. To view a copy of this license, visit http://creativecommons.org/licenses/by-sa/3.0/ or send a letter to Creative Commons, 444 Castro Street, Suite 900, Mountain View, California, 94041, USA.
%
% This work is distributed in the hope that it will be useful, but WITHOUT ANY WARRANTY; without even the implied warranty of MERCHANTABILITY or FITNESS FOR A PARTICULAR PURPOSE.
%
\documentclass[a4paper,12pt, leqno, answers]{exam}
% Customiza\c{c}\~{a}o da classe exam
\newcommand{\mycheader}{Lista 1 - S\'{e}rie de Fourier}
\header{MS560, F560}{\mycheader}{\thepage/\numpages}
\headrule
\footer{Dispon\'{i}vel em \\\input{repository.tex}}{}{Reportar erros para \\\input{maintainer.tex}}
\footrule 
\pagestyle{headandfoot}
\renewcommand{\solutiontitle}{\noindent\textbf{Solu\c{c}\~{a}o:}\enspace}
\SolutionEmphasis{\slshape}
\unframedsolutions
\pointname{}

% Filename: paper_size.tex
%
% This code is part of 'Notas de aula n\~{a}o oficiais de MS650 e F620'
% 
% Description: This file corresponds to the paper size output.
% 
% Created: 14.07.12 11:13:03 AM
% Last Change: 22.08.12 08:24:31 AM
% 
% Authors:
% - Raniere Silva (2012): initial version
% 
% Copyright (c) 2012 Raniere Silva <r.gaia.cs@gmail.com>
% 
% This work is licensed under the Creative Commons Attribution-NonCommercial-NoDerivs 3.0 Unported License. To view a copy of this license, visit http://creativecommons.org/licenses/by-nc-nd/3.0/.
%
% This work is distributed in the hope that it will be useful, but WITHOUT ANY WARRANTY; without even the implied warranty of MERCHANTABILITY or FITNESS FOR A PARTICULAR PURPOSE.
%
% Para impress\~{a}o
\usepackage[top=3cm, bottom=3cm, left=2cm, right=2cm]{geometry}

% Para ereaders (Kindle, Nook, Kobo, ...)
% \usepackage[papersize={160mm,200mm},margin=5mm]{geometry}
% \sloppy

% Para tablets (iPad, GalaxyTab, ...)
% \usepackage[papersize={140mm,190mm},margin=5mm]{geometry}
% \sloppy


% Filename: packages.tex
% 
% This code is part of 'Solutions for MS650, M\'{e}todos de Matem\'{a}tica Aplicada II, and F620, M\'{e}todos Matem\'{a}ticos da F\'{i}sica II'
% 
% Description: This file corresponds to the packages used.
% 
% Created: 07.03.12 04:00:00 PM
% Last Change: 14.07.12 11:03:30 AM
% 
% Authors:
% - Raniere Silva (2012): initial version
% 
% Copyright (c) 2012 Raniere Silva <r.gaia.cs@gmail.com>
% 
% This work is licensed under the Creative Commons Attribution-ShareAlike 3.0 Unported License. To view a copy of this license, visit http://creativecommons.org/licenses/by-sa/3.0/ or send a letter to Creative Commons, 444 Castro Street, Suite 900, Mountain View, California, 94041, USA.
%
% This work is distributed in the hope that it will be useful, but WITHOUT ANY WARRANTY; without even the implied warranty of MERCHANTABILITY or FITNESS FOR A PARTICULAR PURPOSE.
%
\usepackage[utf8]{inputenc}
\usepackage[T1]{fontenc}
\usepackage[brazil]{babel}
\usepackage{amsmath}
\usepackage{amsfonts}
\usepackage{amssymb}
\usepackage{hyperref}
\usepackage{graphicx}
\usepackage{tikz}

% Customiza\c{c}\~{a}o do pacote amsmath
\allowdisplaybreaks[4]

% Novos comandos
\newcommand{\devd}[2]{\frac{\mathrm{d} #1}{\mathrm{d} #2}}
\newcommand{\devdt}[2]{\frac{\mathrm{d}^2 #1}{\mathrm{d} #2^2}}
\newcommand{\devdtm}[3]{\frac{\mathrm{d}^2 #1}{\mathrm{d} #2 \mathrm{d} #3}}
\newcommand{\devp}[2]{\frac{\partial #1}{\partial #2}}
\newcommand{\grad}{\mbox{grad }}
\newcommand{\diver}{\mbox{div }}
\newcommand{\rot}{\mbox{rot }}

\newcommand{\id}[1]{\, \mathrm{d}#1}


\begin{document}
%cover
\thispagestyle{empty}
\input{cover.tex}
\newpage
\setcounter{page}{1}
\begin{questions}
    \question Escreva a s\'{e}rie de Fourier no intervalor $(-\pi, \pi)$ das seguintes fun\c{c}\~{o}es e esboce os gr\'{a}fico das fun\c{c}\~{o}es representadas por essas s\'{e}ries para todo $x$:
    \begin{parts}
        \part $f(x) = \begin{cases}
            -\pi, & - \pi < x < 0, \\
            x, & 0 < x < \pi.
        \end{cases}$
        \begin{solution}
            % TODO Escrever solu\c{c}\~{a}o.
        \end{solution}

        \part $f(x) = | \sin x |$.
        \begin{solution}
            % TODO Escrever solu\c{c}\~{a}o.
        \end{solution}

        \part $f(x) = \begin{cases}
            -x, & -\pi < x < 0, \\
            x, & 0 < x < \pi.
        \end{cases}$
        \begin{solution}
            % TODO Escrever solu\c{c}\~{a}o.
        \end{solution}

        \part $f(x) = \cosh x$.
        \begin{solution}
            % TODO Escrever solu\c{c}\~{a}o.
        \end{solution}

        \part $f(x) = \begin{cases}
            0, & -\pi < x < 0, \\
            x, & 0 < x < \pi/2, \\
            \pi - x, & \pi/2 < x < \pi.
        \end{cases}$
        \begin{solution}
            % TODO Escrever solu\c{c}\~{a}o.
        \end{solution}
    \end{parts}

    \question Escreva a fun\c{c}\~{a}o $f(x) = x^2 / 4$ em s\'{e}rie de Fourier no intervalo $(-\pi, \pi)$ e use o resultado para mostrar que 
    \begin{align*}
        1 + \frac{1}{4} + \frac{1}{9} + \frac{1}{16} + \ldots &= \frac{\pi^2}{6}, \\
        1 - \frac{1}{4} + \frac{1}{9} - \frac{1}{16} + \ldots &= \frac{\pi^2}{12}, \\
        1 + \frac{1}{9} + \frac{1}{25} + \frac{1}{49} + \ldots &= \frac{\pi^2}{8}.
    \end{align*}
    \begin{solution}
        % TODO Escrever solu\c{c}\~{a}o.
    \end{solution}

    \question Escreva as s\'{e}ries de Fourier sobre o intervalo $(-\pi, \pi)$ para as fun\c{c}\~{o}es abaixo:
    \begin{parts}
        \part $f(x) = \begin{cases}
            0, & -\pi < x < 0, \\
            \sin x, & 0 < x < \pi.
        \end{cases}$
        \begin{solution}
            % TODO Escrever solu\c{c}\~{a}o.
        \end{solution}

        \part $f(x) = \exp(x)$.
        \begin{solution}
            % TODO Escrever solu\c{c}\~{a}o.
        \end{solution}
    \end{parts}

    \question Use a representa\c{c}\~{a}o na forma complexa da s\'{e}rie de Fourier para escrever a s\'{e}rie correspondente \`{a} fun\c{c}\~{a}o $f(x) = \exp x$, $-\pi < x < \pi$ e compare esse resultado com o exerc\'{i}cio anterior.
    \begin{solution}
        % TODO Escrever solu\c{c}\~{a}o.
    \end{solution}

    \question Escreva a fun\c{c}ao $f(x) = \left( \pi - x \right) / 2$ em uma s\'{e}rie de Fourier no intervalo $(-\pi, \pi)$ e use o resultado para mostrar que 
    \begin{align*}
        \sum_{n = 1}^\infty \frac{1}{n^2} &= \frac{\pi^2}{6}.
    \end{align*}
    \begin{solution}
        % TODO Escrever solu\c{c}\~{a}o.
    \end{solution}

    \question Escreva a s\'{e}rie de cossenos e a s\'{e}rie de senos de Fourier correspondente \`{a}s fun\c{c}\~{o}es abaixo e esboce o gr\'{a}fico da fun\c{c}\~{a}o representada por essas s\'{e}ries para todo $x$.
    \begin{parts}
        \part $f(x) = 1, 0 < x < \pi$.
        \begin{solution}
            % TODO Escrever solu\c{c}\~{a}o.
        \end{solution}

        \part $f(x) = \pi - x, 0 < x < \pi$.
        \begin{solution}
            % TODO Escrever solu\c{c}\~{a}o.
        \end{solution}

        \part $f(x) = \begin{cases}
            1, & 0 < x < \pi/2, \\
            0, & \pi/2 < x < \pi.
        \end{cases}$
        \begin{solution}
            % TODO Escrver solu\c{c}\~{a}o.
        \end{solution}
    \end{parts}

    \question Escreva a s\'{e}rie de Fourier sobre o intervalo $(0, 2\pi)$ para a fun\c{c}\~{a}o $f(x) = x^2$ e esboce o gr\'{a}fico da fun\c{c}\~{a}o representada por essa s\'{e}rie para todo $x$.
    \begin{solution}
        % TODO Escrver solu\c{c}\~{a}o.
    \end{solution}

    \question Escreva a s\'{e}rie em senos de Fourier sobre o intervalo $(0, 1)$ para a fun\c{c}\~{a}o $f(x) = \cos(\pi x)$.
    \begin{solution}
        % TODO Escrver solu\c{c}\~{a}o.
    \end{solution}
\end{questions}
% \bibliographystyle{plain}
% \bibliography{bibliography}
\end{document}
