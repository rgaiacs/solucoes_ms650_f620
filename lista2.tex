% Filename: lista2.tex
% 
% This code is part of 'Solutions for MS650, M\'{e}todos de Matem\'{a}tica Aplicada II, and F620, M\'{e}todos Matem\'{a}ticos da F\'{i}sica II'
% 
% Description: This file corresponds to the solutions of homework sheet 2.
% 
% Created: 14.07.12 11:13:03 AM
% Last Change: 19.07.12 08:57:01 AM
% 
% Authors:
% - Raniere Silva (2012): initial version
% 
% Copyright (c) 2012 Raniere Silva <r.gaia.cs@gmail.com>
% 
% This work is licensed under the Creative Commons Attribution-ShareAlike 3.0 Unported License. To view a copy of this license, visit http://creativecommons.org/licenses/by-sa/3.0/ or send a letter to Creative Commons, 444 Castro Street, Suite 900, Mountain View, California, 94041, USA.
%
% This work is distributed in the hope that it will be useful, but WITHOUT ANY WARRANTY; without even the implied warranty of MERCHANTABILITY or FITNESS FOR A PARTICULAR PURPOSE.
%
\documentclass[a4paper,12pt, leqno, answers]{exam}
% Customiza\c{c}\~{a}o da classe exam
\newcommand{\mycheader}{Lista 2 - S\'{e}rie de Fourier-Legendre}
\header{MS560, F560}{\mycheader}{\thepage/\numpages}
\headrule
\footer{Dispon\'{i}vel em \\\input{repository.tex}}{}{Reportar erros para \\\input{maintainer.tex}}
\footrule 
\pagestyle{headandfoot}
\renewcommand{\solutiontitle}{\noindent\textbf{Solu\c{c}\~{a}o:}\enspace}
\SolutionEmphasis{\slshape}
\unframedsolutions
\pointname{}

% Filename: paper_size.tex
%
% This code is part of 'Notas de aula n\~{a}o oficiais de MS650 e F620'
% 
% Description: This file corresponds to the paper size output.
% 
% Created: 14.07.12 11:13:03 AM
% Last Change: 22.08.12 08:24:31 AM
% 
% Authors:
% - Raniere Silva (2012): initial version
% 
% Copyright (c) 2012 Raniere Silva <r.gaia.cs@gmail.com>
% 
% This work is licensed under the Creative Commons Attribution-NonCommercial-NoDerivs 3.0 Unported License. To view a copy of this license, visit http://creativecommons.org/licenses/by-nc-nd/3.0/.
%
% This work is distributed in the hope that it will be useful, but WITHOUT ANY WARRANTY; without even the implied warranty of MERCHANTABILITY or FITNESS FOR A PARTICULAR PURPOSE.
%
% Para impress\~{a}o
\usepackage[top=3cm, bottom=3cm, left=2cm, right=2cm]{geometry}

% Para ereaders (Kindle, Nook, Kobo, ...)
% \usepackage[papersize={160mm,200mm},margin=5mm]{geometry}
% \sloppy

% Para tablets (iPad, GalaxyTab, ...)
% \usepackage[papersize={140mm,190mm},margin=5mm]{geometry}
% \sloppy


% Filename: packages.tex
% 
% This code is part of 'Solutions for MS650, M\'{e}todos de Matem\'{a}tica Aplicada II, and F620, M\'{e}todos Matem\'{a}ticos da F\'{i}sica II'
% 
% Description: This file corresponds to the packages used.
% 
% Created: 07.03.12 04:00:00 PM
% Last Change: 14.07.12 11:03:30 AM
% 
% Authors:
% - Raniere Silva (2012): initial version
% 
% Copyright (c) 2012 Raniere Silva <r.gaia.cs@gmail.com>
% 
% This work is licensed under the Creative Commons Attribution-ShareAlike 3.0 Unported License. To view a copy of this license, visit http://creativecommons.org/licenses/by-sa/3.0/ or send a letter to Creative Commons, 444 Castro Street, Suite 900, Mountain View, California, 94041, USA.
%
% This work is distributed in the hope that it will be useful, but WITHOUT ANY WARRANTY; without even the implied warranty of MERCHANTABILITY or FITNESS FOR A PARTICULAR PURPOSE.
%
\usepackage[utf8]{inputenc}
\usepackage[T1]{fontenc}
\usepackage[brazil]{babel}
\usepackage{amsmath}
\usepackage{amsfonts}
\usepackage{amssymb}
\usepackage{hyperref}
\usepackage{graphicx}
\usepackage{tikz}

% Customiza\c{c}\~{a}o do pacote amsmath
\allowdisplaybreaks[4]

% Novos comandos
\newcommand{\devd}[2]{\frac{\mathrm{d} #1}{\mathrm{d} #2}}
\newcommand{\devdt}[2]{\frac{\mathrm{d}^2 #1}{\mathrm{d} #2^2}}
\newcommand{\devdtm}[3]{\frac{\mathrm{d}^2 #1}{\mathrm{d} #2 \mathrm{d} #3}}
\newcommand{\devp}[2]{\frac{\partial #1}{\partial #2}}
\newcommand{\grad}{\mbox{grad }}
\newcommand{\diver}{\mbox{div }}
\newcommand{\rot}{\mbox{rot }}

\newcommand{\id}[1]{\, \mathrm{d}#1}


\begin{document}
%cover
\thispagestyle{empty}
\input{cover.tex}
\newpage
\setcounter{page}{1}
\begin{questions}
    \question Desenvolva a fun\c{c}\~{a}o
    \begin{align*}
        f(x) &= \begin{cases}
            1, & 0 < x < 1, \\
            0, & -1 < x < 0,
        \end{cases}
    \end{align*}
    em uma s\'{e}rie de Fourier-Legendre.
    \begin{solution}
        % TODO Escrever solu\c{c}\~{a}o.
    \end{solution}

    \question Mostre que os coeficientes das expans\~{a}o da fun\c{c}\~{a}o
    \begin{align*}
        f(x) &= x^4 - 3 x^2 + x
    \end{align*}
    em uma s\'{e}rie de Fourier-Legendre s\~{a}o dadas por
    \begin{align*}
        a_0 &= -4/5, \\
        a_1 &= 1, \\
        a_2 &= -10/7, \\
        a_3 &= 0, \\
        a_4 &= 8/35, \\
        a_n &= 0 && n = 5, 6, 7 \ldots
    \end{align*}
    \begin{solution}
        % TODO Escrever solu\c{c}\~{a}o.
    \end{solution}

    \question Seja a s\'{e}rie de Fourier-Legendre de $f(x)$,
    \begin{align*}
        f(x) &= \sum_{n = 0}^\infty a_n P_n(x).
    \end{align*}
    Supondo que essa s\'{e}rie converge uniformemente, mostre que
    \begin{align*}
        \int_{-1}^1 \left[ f(x) \right]^ 2 \,\mathrm{d}x &= \sum_{n = 0}^\infty \frac{2 a_n^2}{2 n + 1}.
    \end{align*}
    \begin{solution}
        % TODO Escrever solu\c{c}\~{a}o.
    \end{solution}

    \question Os polin\^{o}mios de Hermite $H_n(x)$ podem ser definidos pela f\'{o}rmula de Rodrigues,
    \begin{align*}
        H_n(x) &= (-1)^n \exp(x^2) \left[ \frac{\mathrm{d}^n}{\mathrm{d}x^n} \exp(-x^2) \right],
    \end{align*}
    onde $n = 0, 1, 2, \ldots$ e satisfazem a rela\c{c}\~{a}o de ortogonalidade
    \begin{align*}
        \int_{-\infty}^{+\infty} \exp(-x^2) H_n(x) H_m(x) \,\mathrm{d}x &= 2^n n! \sqrt{\pi} \delta_{mn}.
    \end{align*}
    Mostre que os coeficientes do desenvolvimento da fun\c{c}\~{a}o $f(x) = x^3$ em uma s\'{e}rie de Fourier-Hermite s\~{a}o dadas por
    \begin{align*}
        a_0 &= 0, \\
        a_1 &= 3/4, \\
        a_2 &= 0, \\
        a_3 &= 1/8, \\
        a_n &= 0 && n = 4, 5, 6, \ldots
    \end{align*}
    \begin{solution}
        % TODO Escrever solu\c{c}\~{a}o.
    \end{solution}

    \question Mostre que os coeficientes da expans\~{a}o da fun\c{c}\~{a}o $f(x) = x^2$ em uma s\'{e}rie de Fourier-Bessel de ordem zero s\~{a}o dados por
    \begin{align*}
        c_n &= \frac{2 \left( \alpha_n^2 - 4 \right)}{\alpha_n^3 J_1(\alpha_n)}, && n = 1, 2, 3, \ldots
    \end{align*}
    onde $\alpha_n$ \'{e} o $n$-\'{e}simo zero de $J_0(x)$.
    \begin{solution}
        % TODO Escrever solu\c{c}\~{a}o.
    \end{solution}

    \question
    \begin{parts}
        \part Mostre que o problema de Sturm-Liouville singular
        \begin{align*}
            \begin{cases}
                \frac{1}{x} y'' + \frac{1}{x^2} y' = -\lambda y, & 0 < x < 4, \\
                \lim_{x \to 0^+} | y(x) | < \infty, \\
                y(4) = 0,
            \end{cases}
        \end{align*}
        tem autovalores e autofun\c{c}\~{o}es
        \begin{align*}
            \lambda_n &= \left( \frac{3 \alpha_n}{16} \right)^2, & y_n(x) & J_0\left( \frac{\alpha_n x^{3/2}}{8} \right),
        \end{align*}
        onde $n = 1, 2, 3, \ldots$ e $\alpha_n$ \'{e} o $n$-\'{e}simo zero da fun\c{c}\~{a}o de Bessel de primeira esp\'{e}cie e ordem zero.
        \begin{solution}
            % TODO Escrever solu\c{c}\~{a}o.
        \end{solution}

        \part Mostre que a expans\~{a}o da fun\c{c}\~{a}o $f(x) = 1$ em termo dessa autofun\c{c}\~{a}o \'{e} dada por
        \begin{align*}
            1 &= \sum_{n = 1}^\infty \frac{2}{\alpha_n J_1(\alpha_n)} y_n(x).
        \end{align*}
        \begin{solution}
            % TODO EScrever solu\c{c}\~{a}o.
        \end{solution}
    \end{parts}

    \question[P1 de 2006] Seja a s\'{e}rie de Fourier-Bessel
    \begin{align*}
        f(x) &= \sum_{n = 1}^\infty c_n J_k(\lambda_{kn} x/a),
    \end{align*}
    onde $\lambda_{kn}$ \'{e} o $n$-\'{e}simo zero de $J_k(x)$ e $0 < x < a$.
    \begin{parts}
        \part Supondo a converg\^{e}ncia uniforme, mostre que a identidade de Parseval para essa s\'{e}rie \'{e}
        \begin{align*}
            \int_0^a \left[ f(x) \right]^2 x \id{x} &= \frac{a^2}{2} \sum_{n = 1}^\infty c_n^2 \left[ J_{k+1}\left( \lambda_{kn} \right) \right]^2.
        \end{align*}
        \begin{solution}
            Temos que
            \begin{align*}
                \int_0^a \left[ f(x) \right]^2 x \id{x} &= \int_0^a x \left[ \sum_{n = 1}^\infty \sum_{m = 1}^\infty c_n c_m J_k\left( \frac{\lambda_{kn} x}{a} J_k\left( \frac{\lambda_{km} x}{a} \right) \right) \right] \id{x} \\
                &= \sum_{n = 1} \sum_{m = 1} c_n c_m \int_0^a x J_k\left( \frac{\lambda_{kn} x}{a} \right) J_k\left( \frac{\lambda_{km} x}{a} \right) \id{x} \\
                &= \sum_{n = 1}^\infty \sum_{m = 1}^\infty c_n c_m \frac{a^2}{2} \left[ J_{k + 1}\left( \lambda_{kn} \right) \right]^2 \delta_{nm} \\
                &= \frac{a^2}{2} \sum_{n = 1}^\infty c_n^2 \left[ J_{k+1}\left( \lambda_{kn} \right) \right]^2.
            \end{align*}
        \end{solution}

        \part Sabendo que o desenvolvimento de $f(x) = x^k$ em termos dessa s\'{e}rie \'{e} dado por
        \begin{align*}
            x^k = \sum{n = 1}^\infty \frac{2 a^k J_k\left( \lambda_{kn} x / a \right)}{\lambda_{kn} J_{k+1}\left( \lambda_{kn} \right)},
        \end{align*}
        use essa identidade para mostrar que
        \begin{align*}
            \sum_{n = 1}^\infty \frac{1}{\lambda_{kn}^2} = \frac{1}{4\left( k + 1 \right)}.
        \end{align*}
        \begin{solution}
            Temos que
            \begin{align*}
                x^k = \sum_{n = 1}^\infty \frac{2 a^k J_k\left( \lambda_{kn} x / a \right)}{\lambda_{kn} J_{k+1}\left( \lambda_{kn} \right)}
            \end{align*}
            que implica em
            \begin{align*}
                \begin{cases}
                    f(x) = x^k, \\
                    c_n = \left( 2 a^k \right) / \left[ \lambda_{kn} J_{k+1}\left( \lambda_{kn} \right) \right].
                \end{cases}
            \end{align*}
            Logo,
            \begin{align*}
                \int_0^a \left[ f(x) \right]^2 x \id{x} &= \int_0^a x^{2k + 1} \id{x} \\
                &= \left. \frac{x^{2k + 2}}{2k + 2} \right|_0^a \\
                &= \frac{a^{2k + 2}}{2k + 2}, \\
                \frac{a^2}{2} \sum_{n = 1}^\infty c_n^2 \left[ J_{k+1}\left( \lambda_{kn} \right) \right]^2 &= \frac{a^2}{2} \sum_{n = 1}^\infty \frac{4 a^{2k} \left[ J_{k+1}\left( \lambda_{kn} \right) \right]^2}{\left( \lambda_{kn} \right)^2 \left[ J_{k+1}\left( \lambda_{kn} \right) \right]^2} \\
                &= 2 a^{2k + 2} \sum_{n = 1}^\infty \frac{1}{\lambda_{kn}^2}.
            \end{align*}
            Portanto,
            \begin{align*}
                \frac{a^{2k + 2}}{2k + 2} &= 2 a^{2k + 2} \sum_{n = 1}^\infty \frac{1}{\lambda_{kn}^2}
            \end{align*}
            e
            \begin{align*}
                \sum_{n = 1}^\infty \frac{1}{\lambda_{kn}^2} &= \frac{1}{4 \left( k + 1 \right)}.
            \end{align*}
        \end{solution}
    \end{parts}

    \question[P1 de 2006] Seja $f(x) = \text{sign}(x)$,
    \begin{align*}
        \text{sign}(x) = \begin{cases}
            1, & x > 0, \\
            -1, & x < 0.
        \end{cases}
    \end{align*}
    Mostre que seu desenvolvimento em uma s\'{e}rie de Fourier-Legendre \'{e} dado por
    \begin{align*}
        \text{sign}(x) &= \sum_{n = 0}^\infty \frac{(-1)^n (1/2)_n (2n + 3/2)}{(n + 1)!} P_{2n + 1}(x),
    \end{align*}
    onde $(a)_n = a (a + 1) \cdots (a + n - 1)$ \'{e} o s\'{i}mbolo de Pochhammer e $P_n(x)$ \'{e} o $n$-\'{e}simo polini\^{o}mio de Legendre.
    \begin{solution}
        Temos que
        \begin{align*}
            f(x) &= \text{sign}(x) = \sum_{n = 0}^\infty a_n P_n(x)
        \end{align*}
        onde
        \begin{align*}
            a_n = \frac{2n + 1}{2} \int_{-1}^1 \text{sign}(x) P_n(x) \id{x}.
        \end{align*}

        Ent\~{a}o
        \begin{align*}
            \int_{-1}^1 \text{sign}(x) P_n(x) \id{x} &= \int_{-1}^0 (-1) P_n(x) \id{x} + \int_0^1 (1) P_n(x) \id{x} \\
            &= \int_1^0 P_n(-x) \id{X} + \int_0^1 P_n(x) \id{x} \\
            &= -(-1)^n \int_0^1 P_n(x) \id{x} + \int_0^1 P_n(x) \id{x}
        \end{align*}
        e portanto
        \begin{align*}
            a_{2k} &= 0, \\
            a_{2k+1} &= \frac{4k + 3}{2} 2 \int_0^1 P_{2k+1}(x) \id{x}.
        \end{align*}

        Usando $P'_{n + 1}(x) - P'_{n-1}(x) = \left( 2n + 1 \right) P_n(x)$ temos
        \begin{align*}
            \int_0^1 P_{2k + 1}(x) \id{x} &= \frac{1}{4k + 3} \int_0^1 \left( P'_{2k + 2}(x) - P'_{2k}(x) \right) \id{x} \\
            &= \frac{1}{4k + 3} \left[ \left. P_{2k+2}(x) \right|_0^1 - \left. P_{2k}(x) \right|_0^1 \right] \\
            &= \frac{1}{4k + 3} \left[ P_{2k}(0) - P_{2k + 2}(0) \right] \\
            &= \frac{1}{4k + 3} \left[ \frac{(-1)^k \left( 1/2 \right)_k}{k!} - \frac{(-1)^k \left( 1/2 \right)_{k + 1}}{\left( k + 1 \right)!} \right] \\
            &= \frac{1}{4k + 3} \frac{(-1)^k \left( 1/2 \right)_k}{k!} \left[ 1 - \frac{\left( 1/2 + k \right)}{k + 1} \right] \\
            &= \frac{(-1)^k \left( 1/2 \right)_k \left( 2k + 3/2 \right)}{\left( 4k + 3 \right) \left( k + 1 \right)!}
        \end{align*}
        e portanto
        \begin{align*}
            a_{2k + 1} &= \frac{4 k + 3}{2} 2 \frac{(-1)^k \left( 1/2 \right)_k \left( 2k + 3/2 \right)}{\left( 4k + 3 \right) \left( k + 1 \right)!} \\
            &= \frac{(-1)^k \left( 1/2 \right)_k \left( 2k + 3/2 \right)}{\left( k + 1 \right)!}.
        \end{align*}
        
        Por fim,
        \begin{align*}
            \text{sign}(x) &= \sum_{k = 0}^\infty \frac{(-1)^k \left( 1/2 \right)_k \left( 2k + 3/2 \right)}{\left( k + 1 \right)!} P_{2k + 1}(x).
        \end{align*}
    \end{solution}

    \question[T2 de 2011, P1 de 2011] Sejam $L_n(x)$ ($n = 0, 1, 2, \ldots$) os polinîmios de Laguerre. Mostre que o desenvolvimento da fun\c{c}\~{a}o $f(x) = \exp(-ax)$ ($a > 0$) em uma s\'{e}rie de Fourier-Laguerre pode ser escrito na forma
    \begin{align*}
        \exp\left( -ax \right) &= \frac{1}{1 + a} \sum_{n = 0}^\infty \left( \frac{a}{1 + a} \right)^n L_n(x),
    \end{align*}
    onde $0 \leq x \leq \infty$.
    \begin{solution}
        Sabemos que
        \begin{align*}
            \exp\left( -ax \right) &= \sum_{n = 0}^\infty c_n L_n(x)
        \end{align*}
        onde
        \begin{align*}
            c_n &= \frac{<\exp(-ax), L_n>}{\| L_n \|^2} = \int_0^\infty \exp(-x) \exp(-ax) L_n(x) \id{x}.
        \end{align*}
        Usando que
        \begin{align*}
            L_n(x) &= \frac{\exp(x)}{n!} \frac{\id{}^n}{\id{x^n}}\left( \exp(-x) x^n \right)
        \end{align*}
        temos que
        \begin{align*}
            c_n &= \int_0^\infty \exp(-x) \exp(-ax) \frac{\exp(x)}{n!} \frac{\id{}^n}{\id{x^n}}\left( \exp(-x) x^n \right) \id{x} \\
            &= \frac{1}{n!} \int_0^\infty \exp\left( -ax \right) \frac{\id{}^n}{\id{x^n}}\left( \exp(-x) x^n \right) \id{x}
        \end{align*}
        % TODO Terminar de escrever a solu\c{c}\~{a}o.
    \end{solution}

    \question[P1 de 2011] Os polin\^{o}mios de Hermite $H_n(x)$ podem ser definidos pela f\'{o}rmula de Rodrigues,
    \begin{align*}
        H_n(x) &= (-1)^n \exp(x^2) \frac{\id{}^n}{\id{x^n}}\left( \exp(-x^2) \right),
    \end{align*}
    onde $n = 0, 1, 2, \ldots$ e satisfazem a rela\c{c}\~{a}o de ortogonalidade
    \begin{align*}
        \int_{-\infty}^\infty \exp(-x^2) H_n(x) H_m(x) \id{x} &= 2^n n! \sqrt{\pi} \delta_{mn}.
    \end{align*}
    Encontre o desenvolvimento da fun\c{c}\~{a}o
    \begin{align*}
        f(x) = x^4
    \end{align*}
    em uma s\'{e}rie de Fourier-Hermite.
    \begin{solution}
        % TODO Escrever a solu\c{c}\~{a}o.
    \end{solution}
\end{questions}
% \bibliographystyle{plain}
% \bibliography{bibliography}
\end{document}
