% Filename: lista3.tex
% 
% This code is part of 'Solutions for MS650, Métodos de Matemática
% Aplicada II, and F620, Métodos Matemáticos da F\'{i}sica II'
% 
% Description: This file corresponds to the solutions of homework sheet 3.
% 
% Created: 14.07.12 11:13:03 AM Last Change: 14.07.12 11:13:03 AM
% 
% Authors:
% - Raniere Silva (2012): initial version
% 
% Copyright (c) 2012 Raniere Silva <r.gaia.cs@gmail.com>
% 
% This work is licensed under the Creative Commons Attribution-ShareAlike 3.0
% Unported License. To view a copy of this license, visit
% http://creativecommons.org/licenses/by-sa/3.0/ or send a letter to Creative
% Commons, 444 Castro Street, Suite 900, Mountain View, California, 94041, USA.
%
% This work is distributed in the hope that it will be useful, but WITHOUT ANY
% WARRANTY; without even the implied warranty of MERCHANTABILITY or FITNESS FOR
% A PARTICULAR PURPOSE.
%
\documentclass[a4paper,12pt, leqno, answers]{exam}
% Customização da classe exam
\newcommand{\mycheader}{Lista 3 - Transformada de Fourier}
\header{MS560, F560}{\mycheader}{\thepage/\numpages}
\headrule
\footer{Dispon\'{i}vel em \\\input{repository.tex}}{}{Reportar erros para
\\\input{maintainer.tex}}
\footrule 
\pagestyle{headandfoot}
\renewcommand{\solutiontitle}{\noindent\textbf{Solução:}\enspace}
\SolutionEmphasis{\slshape}
\unframedsolutions
\pointname{}

% Filename: paper_size.tex
%
% This code is part of 'Notas de aula n\~{a}o oficiais de MS650 e F620'
% 
% Description: This file corresponds to the paper size output.
% 
% Created: 14.07.12 11:13:03 AM
% Last Change: 22.08.12 08:24:31 AM
% 
% Authors:
% - Raniere Silva (2012): initial version
% 
% Copyright (c) 2012 Raniere Silva <r.gaia.cs@gmail.com>
% 
% This work is licensed under the Creative Commons Attribution-NonCommercial-NoDerivs 3.0 Unported License. To view a copy of this license, visit http://creativecommons.org/licenses/by-nc-nd/3.0/.
%
% This work is distributed in the hope that it will be useful, but WITHOUT ANY WARRANTY; without even the implied warranty of MERCHANTABILITY or FITNESS FOR A PARTICULAR PURPOSE.
%
% Para impress\~{a}o
\usepackage[top=3cm, bottom=3cm, left=2cm, right=2cm]{geometry}

% Para ereaders (Kindle, Nook, Kobo, ...)
% \usepackage[papersize={160mm,200mm},margin=5mm]{geometry}
% \sloppy

% Para tablets (iPad, GalaxyTab, ...)
% \usepackage[papersize={140mm,190mm},margin=5mm]{geometry}
% \sloppy


% Filename: packages.tex
% 
% This code is part of 'Solutions for MS650, M\'{e}todos de Matem\'{a}tica Aplicada II, and F620, M\'{e}todos Matem\'{a}ticos da F\'{i}sica II'
% 
% Description: This file corresponds to the packages used.
% 
% Created: 07.03.12 04:00:00 PM
% Last Change: 14.07.12 11:03:30 AM
% 
% Authors:
% - Raniere Silva (2012): initial version
% 
% Copyright (c) 2012 Raniere Silva <r.gaia.cs@gmail.com>
% 
% This work is licensed under the Creative Commons Attribution-ShareAlike 3.0 Unported License. To view a copy of this license, visit http://creativecommons.org/licenses/by-sa/3.0/ or send a letter to Creative Commons, 444 Castro Street, Suite 900, Mountain View, California, 94041, USA.
%
% This work is distributed in the hope that it will be useful, but WITHOUT ANY WARRANTY; without even the implied warranty of MERCHANTABILITY or FITNESS FOR A PARTICULAR PURPOSE.
%
\usepackage[utf8]{inputenc}
\usepackage[T1]{fontenc}
\usepackage[brazil]{babel}
\usepackage{amsmath}
\usepackage{amsfonts}
\usepackage{amssymb}
\usepackage{hyperref}
\usepackage{graphicx}
\usepackage{tikz}

% Customiza\c{c}\~{a}o do pacote amsmath
\allowdisplaybreaks[4]

% Novos comandos
\newcommand{\devd}[2]{\frac{\mathrm{d} #1}{\mathrm{d} #2}}
\newcommand{\devdt}[2]{\frac{\mathrm{d}^2 #1}{\mathrm{d} #2^2}}
\newcommand{\devdtm}[3]{\frac{\mathrm{d}^2 #1}{\mathrm{d} #2 \mathrm{d} #3}}
\newcommand{\devp}[2]{\frac{\partial #1}{\partial #2}}
\newcommand{\grad}{\mbox{grad }}
\newcommand{\diver}{\mbox{div }}
\newcommand{\rot}{\mbox{rot }}

\newcommand{\id}[1]{\, \mathrm{d}#1}

\DeclareMathOperator{\sgn}{sgn}

\begin{document}
%cover
\thispagestyle{empty}
\input{cover.tex}
\newpage
\setcounter{page}{1}
\begin{questions}
    \question Mostre que a transformada de Fourier das funções $f(x)$
    abaixo são dadas pelas correspondentes funções $F(k)$ onde
    $\sgn(x) = 2 H(x) - 1$ é a função sinal (ou seja, $\sgn(x) = 1$
    se $x > 0$ e $\sgn(x) = -1$ se $x < 0$) e $H(x)$ é a função
    escada (ou seja, $H(x) = 1$ se $x > 0$ e $H(x) = 0$ se $x < 0$).
    \begin{center}
        \begin{tabular}{|c|c|}
            \hline
            $f(x)$ & $F(x)$ \\ \hline
            $(1 - x^2) H(1 - |x|)$ & $\left( -1 / \sqrt{2 \pi} \right) \left(
            4 / k^3 \right) \left( k \cos(k) - \sin(k) \right)$ \\ \hline
            $\sin\left( \alpha x \right)$ & $i \sqrt{\pi / 2} \left( \delta(k -
            \alpha) - \delta(k + \alpha) \right)$ \\ \hline
            $1 / x$ & $i \sqrt{\pi/2} \sgn(k) = i \sqrt{2 \pi} \left( H(k) - 1/2
            \right)$ \\ \hline
            $H(x)$ & $ \left( 1 / \sqrt{2 \pi} \right) \left( \pi \delta(k) + 1
            / k \right)$ \\ \hline
        \end{tabular}
    \end{center}
    \begin{solution}
        % TODO Escrever solução.
    \end{solution}

    \question Usando $\delta(x) = H'(x)$ e a expressão acima para a
    transformada de Fourier de $H(x)$, mostre que $\mathcal{F}[\delta(x)] = 1 /
    \sqrt{2 \pi}$.
    \begin{solution}
        % TODO Escrever solução.
    \end{solution}

    \question Mostre que
    \begin{align*}
        \int_0^\infty \left( \frac{x \cos(x) - \sin(x)}{x^3} \right) \cos(x/2)
        \id{x} &= \frac{3 \pi}{16}.
    \end{align*}
    \begin{solution}
        % TODO Escrever solução.
    \end{solution}

    \question Use a f\'{o}rmula integral de Fourier para mostrar que
    \begin{parts}
        \part $\int_0^\infty \left[ \left( \cos(xy) \right) / \left( 1 + y^2
        \right) \right] \id{x} = \left( 2 / \pi \right) \exp(-|x|),$
        \begin{solution}
            % TODO Escrever solução.
        \end{solution}

        \part $\int_0^\infty \left[ \left( y \sin(xy) \right) / \left( 1 + y^2
        \right) \right] \id{y} = \left( 2/\pi \right) \sgn(x) \exp(-|x|)$.
        \begin{solution}
            % TODO Escrever solução.
        \end{solution}
    \end{parts}

    \question Use a identidade de Parseval para mostrar que
    \begin{parts}
        \part $\int_0^\infty \left[ 1 / \left( 1 + x^2 \right)^2 \right] \id{x}
        = \pi / 4$,
        \begin{solution}
            % TODO Escrever solução.
        \end{solution}

        \part $\int_0^\infty \left[ \left( x \cos(x) - \sin(x) \right)^2 / x^6
        \right] \id{x} = \pi / 15$.
        \begin{solution}
            % TODO Escrever solução.
        \end{solution}
    \end{parts}

    \question Verifique a validade da convolução para as funções
    \begin{align*}
        f(x) = g(x) &= \begin{cases}
            1, & |x| < 1, \\
            0, & |x| > 1.
        \end{cases}
    \end{align*}
    \begin{solution}
        % TODO Escrever solução.
    \end{solution}

    \question Ilustre o Princ\'{i}pio de Incerteza de Heisenberg para a
    função
    \begin{align*}
        f(x) &= \frac{a}{x^2 + a^2},
    \end{align*}
    com $a > 0$, mostrando que nesse caso $\Delta x = a$ e $\Delta k = 1 /
    \sqrt{2} a$.
    \begin{solution}
        % TODO Escrever solução.
    \end{solution}

    \question Seja $\psi(x, t)$ uma função dada na forma
    \begin{align*}
        \psi(x, t) = \int_{-\infty}^\infty \phi(k) \exp\left( i (k x - k^2 t / 2
        \right) \id{k}.
    \end{align*}
    Sabendo que
    \begin{align*}
        \psi(x, 0) &= \exp(-x^2 / 2),
    \end{align*}
    mostre que
    \begin{align*}
        \psi(x, t) &= \frac{1}{(1 + i t)^{1/2}} \exp\left( -x^2 / \left[ 2 (1 +
        i t) \right] \right).
    \end{align*}

    \question[T3 de 2011] Seja $f(x)$ dada por
    \begin{align*}
        f(x) &= \begin{cases}
            1 - \left( 1/2 \right) |x|, & |x| \leq 2, \\
            0, & |x| > 2.
        \end{cases}
    \end{align*}
    \begin{parts}
        \part Mostre que a transformada de Fourier $F(k)$ de $f(x)$ é dada
        por
        \begin{align*}
            F(k) &= \sqrt{\frac{2}{\pi}} \left( \frac{\sin(k)}{k} \right)^2.
        \end{align*}
        \begin{solution}
            Temos que
            \begin{align*}
                F(k) &= \frac{1}{\sqrt{2 \pi}} \int_{-\infty}^\infty f(x)
                \exp\left( i k x \right) \id{x} \\
                &= \frac{1}{\sqrt{2 \pi}} \int_{-2}^2 \left( 1 - |x| / 2 \right)
                \exp\left( i k x \right) \id{x} \\
                \begin{split}
                    &= \frac{1}{\sqrt{2 \pi}} \left[ \int_{-2}^0 \left( 1 + x / 2
                    \right) \exp\left( i k x \right) \id{x} \right. \\
                    &\quad \left. {}+ \int_0^2 \left( 1 - x
                    / 2 \right) \exp\left( i k x \right) \id{x} \right] \\
                \end{split} \\
                \begin{split}
                    &= \frac{1}{\sqrt{2 \pi}} \left[ \left. \left(
                    \frac{\exp(i k x)}{i k} + \frac{x}{2} \frac{\exp(i k x}{i k}
                    - \frac{\exp(i k x)}{2 (i k)^2} \right) \right|_{-2}^0
                    \right. \\
                    &\quad \left. {}+ \left. \left( \frac{\exp(i k x}{i k} -
                    \frac{x}{2} \frac{\exp(i k x)}{i k} + \frac{\exp(i k x}{2 (i
                    k)^2} \right) \right|_0^2 \right]
                \end{split} \\
                &= \frac{1}{2 \sqrt{2 \pi}} \left[ \left( \frac{\exp(i k)}{i k}
                \right)^2 - \frac{2}{(i k)^2} + \left( \frac{\exp(-i k)}{i k}
                \right)^2 \right] \\
                &= \frac{1}{2 \sqrt{2 \pi}} \left( \frac{\exp(i k) + \exp(-i
                k)}{i k} \right)^2 \\
                &= \sqrt{\frac{2}{\pi}} \left( \frac{\sin(k)}{k} \right)^2.
            \end{align*}
        \end{solution}

        \part Use a identidade de Parseval para calcular a integral
        \begin{align*}
            \int_{-\infty}^\infty \left( \frac{\sin(k)}{k} \right)^4 \id{k}.
        \end{align*}
        \begin{solution}
            Temos, pela identidade de Parseval, que
            \begin{align*}
                \int_{-\infty}^\infty |f(x)|^2 \id{x} &= \int_{-\infty}^\infty
                |F(k)|^2 \id{k}.
            \end{align*}
            Logo,
            \begin{align*}
                \int_{-\infty}^\infty \left( \frac{\sin(k)}{k} \right)^4 \id{k}
                &= \frac{\pi}{2} \int_{-\infty}^\infty \left( F(k) \right)^2
                \id{k} \\
                &= \frac{\pi}{2} \int_{-\infty}^\infty \left( f(x) \right)^2
                \id{x} \\
                &= \frac{\pi}{2} \left[ \int_{-2}^0 \left( 1 +
                \frac{x}{2} \right)^2 \id{x} + \int_0^2 \left( 1 -
                \frac{x}{2} \right)^2 \id{x} \right] \\
                \begin{split}
                    &= \frac{\pi}{2} \left[ \left. \left( x +
                    \frac{x^2}{2} + \frac{x^3}{12} \right) \right|_{-2}^0
                    \right. \\
                    &\quad \left. {}+ \left. \left( x - \frac{x^2}{12} +
                    \frac{x^3}{12} \right) \right|_0^2 \right]
                \end{split} \\
                &= \frac{\pi}{2} \frac{16}{12} \\
                &= 2 \pi / 3.
            \end{align*}
        \end{solution}
    \end{parts}

    \question[P1 de 2011]
    \begin{parts}
        \part Mostre que a transformada de Fourier de $f(x) = \exp(-a x^2)$
        é dada por
        \begin{align*}
            F(k) &= \frac{1}{\sqrt{2 a}} \exp\left( -k^2 / (4a) \right).
        \end{align*}
        \begin{solution}
            Temos que
            \begin{align*}
                F(k) &= \frac{1}{\sqrt{2 \pi}} \int_{-\infty}^\infty \exp\left(
                -a x^2 \right) \exp\left( i k x \right) \id{x} \\
                &= \frac{1}{\sqrt{2 \pi}} \int_{-\infty}^\infty \exp\left[ -
                \left( \sqrt{a}x - i k / (2 \sqrt{a}) \right)^2 \exp\left( - k^2
                / (4 a) \right) \right] \id{x} \\
                &= \frac{\exp\left( -k^2 / (4a) \right)}{\sqrt{2 \pi}}
                \int_{-\infty}^\infty \exp\left[ -\left( \sqrt{a} x - i k /
                (2\sqrt{a}) \right)^2 \right] \id{x} \\
                &= \frac{\exp\left( -k^2 / 4a \right)}{\sqrt{2 \pi}}
                \frac{1}{\sqrt{a}} \int_{-\infty}^\infty \exp\left( -u^2
                \right) \id{u} \\
                &= \frac{\exp(-k^2 / (4a)}{\sqrt{2 a}}.
            \end{align*}
        \end{solution}

        \part Discuta o comportamente de $f(x)$ e $F(k)$ para diferentes valores
        de $a$ e no limite em que $a \to 0$.
        \begin{solution}
            Temos que $\lim_{a \to 0} f(x) = \exp(0) = 1$ e
            \begin{align*}
                F[1] &= \frac{1}{\sqrt{2\pi}} \int_{-\infty}^\infty \exp(i k x)
                \id{x} \\
                &= \sqrt{2 \pi} \frac{1}{2 \pi} \int_{-\infty}^\infty \exp(i k
                x) \id{x} \\
                &= \sqrt{2 \pi} \delta(k).
            \end{align*}
            Logo, $\lim_{a \to 0}\left( \exp(-k^2 / (4a)) / \sqrt{2a} \right) =
            \sqrt{2 \pi} \delta(k)$.
        \end{solution}
    \end{parts}
\end{questions}
% \bibliographystyle{plain}
% \bibliography{bibliography}
\end{document}
