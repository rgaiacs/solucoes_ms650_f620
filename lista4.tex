% Filename: lista4.tex
% 
% This code is part of 'Solutions for MS650, M\'{e}todos de Matem\'{a}tica
% Aplicada II, and F620, M\'{e}todos Matem\'{a}ticos da F\'{i}sica II'
% 
% Description: This file corresponds to the solutions of homework sheet 4.
% 
% Created: 14.07.12 11:13:03 AM Last Change: 14.07.12 11:13:03 AM
% 
% Authors:
% - Raniere Silva (2012): initial version
% 
% Copyright (c) 2012 Raniere Silva <r.gaia.cs@gmail.com>
% 
% This work is licensed under the Creative Commons Attribution-ShareAlike 3.0
% Unported License. To view a copy of this license, visit
% http://creativecommons.org/licenses/by-sa/3.0/ or send a letter to Creative
% Commons, 444 Castro Street, Suite 900, Mountain View, California, 94041, USA.
%
% This work is distributed in the hope that it will be useful, but WITHOUT ANY
% WARRANTY; without even the implied warranty of MERCHANTABILITY or FITNESS FOR
% A PARTICULAR PURPOSE.
%
\documentclass[a4paper,12pt, leqno, answers]{exam}
% Customiza\c{c}\~{a}o da classe exam
\newcommand{\mycheader}{Lista 4 - S\'{e}rie de Fourier (2)}
\header{MS560, F560}{\mycheader}{\thepage/\numpages}
\headrule
\footer{Dispon\'{i}vel em \\\input{repository.tex}}{}{Reportar erros para
\\\input{maintainer.tex}}
\footrule 
\pagestyle{headandfoot}
\renewcommand{\solutiontitle}{\noindent\textbf{Solu\c{c}\~{a}o:}\enspace}
\SolutionEmphasis{\slshape}
\unframedsolutions
\pointname{}

% Filename: paper_size.tex
%
% This code is part of 'Notas de aula n\~{a}o oficiais de MS650 e F620'
% 
% Description: This file corresponds to the paper size output.
% 
% Created: 14.07.12 11:13:03 AM
% Last Change: 22.08.12 08:24:31 AM
% 
% Authors:
% - Raniere Silva (2012): initial version
% 
% Copyright (c) 2012 Raniere Silva <r.gaia.cs@gmail.com>
% 
% This work is licensed under the Creative Commons Attribution-NonCommercial-NoDerivs 3.0 Unported License. To view a copy of this license, visit http://creativecommons.org/licenses/by-nc-nd/3.0/.
%
% This work is distributed in the hope that it will be useful, but WITHOUT ANY WARRANTY; without even the implied warranty of MERCHANTABILITY or FITNESS FOR A PARTICULAR PURPOSE.
%
% Para impress\~{a}o
\usepackage[top=3cm, bottom=3cm, left=2cm, right=2cm]{geometry}

% Para ereaders (Kindle, Nook, Kobo, ...)
% \usepackage[papersize={160mm,200mm},margin=5mm]{geometry}
% \sloppy

% Para tablets (iPad, GalaxyTab, ...)
% \usepackage[papersize={140mm,190mm},margin=5mm]{geometry}
% \sloppy


% Filename: packages.tex
% 
% This code is part of 'Solutions for MS650, M\'{e}todos de Matem\'{a}tica Aplicada II, and F620, M\'{e}todos Matem\'{a}ticos da F\'{i}sica II'
% 
% Description: This file corresponds to the packages used.
% 
% Created: 07.03.12 04:00:00 PM
% Last Change: 14.07.12 11:03:30 AM
% 
% Authors:
% - Raniere Silva (2012): initial version
% 
% Copyright (c) 2012 Raniere Silva <r.gaia.cs@gmail.com>
% 
% This work is licensed under the Creative Commons Attribution-ShareAlike 3.0 Unported License. To view a copy of this license, visit http://creativecommons.org/licenses/by-sa/3.0/ or send a letter to Creative Commons, 444 Castro Street, Suite 900, Mountain View, California, 94041, USA.
%
% This work is distributed in the hope that it will be useful, but WITHOUT ANY WARRANTY; without even the implied warranty of MERCHANTABILITY or FITNESS FOR A PARTICULAR PURPOSE.
%
\usepackage[utf8]{inputenc}
\usepackage[T1]{fontenc}
\usepackage[brazil]{babel}
\usepackage{amsmath}
\usepackage{amsfonts}
\usepackage{amssymb}
\usepackage{hyperref}
\usepackage{graphicx}
\usepackage{tikz}

% Customiza\c{c}\~{a}o do pacote amsmath
\allowdisplaybreaks[4]

% Novos comandos
\newcommand{\devd}[2]{\frac{\mathrm{d} #1}{\mathrm{d} #2}}
\newcommand{\devdt}[2]{\frac{\mathrm{d}^2 #1}{\mathrm{d} #2^2}}
\newcommand{\devdtm}[3]{\frac{\mathrm{d}^2 #1}{\mathrm{d} #2 \mathrm{d} #3}}
\newcommand{\devp}[2]{\frac{\partial #1}{\partial #2}}
\newcommand{\grad}{\mbox{grad }}
\newcommand{\diver}{\mbox{div }}
\newcommand{\rot}{\mbox{rot }}

\newcommand{\id}[1]{\, \mathrm{d}#1}

\DeclareMathOperator{\erfc}{erfc}

\begin{document}
%cover
\thispagestyle{empty}
\input{cover.tex}
\newpage
\setcounter{page}{1}
\begin{questions}
    \question Seja a equa\c{c}\~{a}o
    \begin{align*}
        \devp{^2 G(x, \epsilon)}{x^2} - \omega^2 G(x, \epsilon) &= \delta(x -
        \epsilon),
    \end{align*}
    onde $-\infty < x$, $\epsilon < \infty$ e $\omega > 0$. Use o m\'{e}todo da
    transformada de Fourier para mostrar que $G(x, \epsilon)$ pode ser escrita
    na forma
    \begin{align*}
        G(x, \epsilon) &= \frac{-1}{2\pi} \int_{-\infty}^\infty
        \frac{\exp\left( -i k (x - \epsilon) \right)}{\omega^2 + k^2} \id{k} =
        \frac{-1}{2 \omega} \exp\left( -\omega |x - \epsilon| \right).
    \end{align*}
    \begin{solution}
        % TODO Escrever solu\c{c}\~{a}o.
    \end{solution}

    \question Usando a transformada de Fourier, mostre que a solu\c{c}\~{a}o da
    equa\c{c}\~{a}o
    \begin{align*}
        \kappa^2 \devd{^4 y}{x^4} + y &= \delta(x),
    \end{align*}
    onde $-\infty < x < \infty$, satisfazendo as condi\c{c}\~{o}es $\lim_{x \to
    \pm\infty} y^{(k)} = 0$ ($k = 0, 1, 2, 3, 4$) \'{e} dado por
    \begin{align*}
        y(x) &= \frac{1}{2 \sqrt{2\kappa}} \left( \cos\left(
        \frac{|x|}{\sqrt{2\kappa}} \right) + \sin\left(
        \frac{|x|}{\sqrt{2\kappa}} \right) \right) \exp\left( -|x|
        \sqrt{2\kappa} \right).
    \end{align*}
    \begin{solution}
        % TODO Escrever solu\c{c}\~{a}o.
    \end{solution}

    \question Usando a transformada de Fourier, mostre que a solu\c{c}\~{a}o da
    equa\c{c}\~{a}o
    \begin{align*}
        \devp{\omega}{t} &= -\kappa \devp{\omega}{x} + D \devp{^2 \omega}{x^2},
    \end{align*}
    onde $t > 0$, $-\infty < x < \infty$, satisfazendo as condi\c{c}\~{o}es que
    $\omega(x, t)$, $\left[ partial \omega(x, t) \right] / \partial x$ e $\left[
    \partial^2 \omega(x, t) \right] / \partial x^2$ se anulem para $x \to \pm
    \infty$ e a condi\c{c}\~{a}o inicial
    \begin{align*}
        \omega(x, 0) &= \delta(x - x_0)
    \end{align*}
    \'{e} dada por
    \begin{align*}
        \omega(x, t) &= \frac{1}{\sqrt{4 \pi D t}} \exp\left( -
        \frac{(x - x_0 - \kappa t)^2}{4 D t} \right).
    \end{align*}
    \begin{solution}
        % TODO Escrever solu\c{c}\~{a}o.
    \end{solution}

    \question Usando o m\'{e}todo da transformada de Fourier, mostre que a
    solu\c{c}\~{a}o da equa\c{c}\~{a}o integral
    \begin{align*}
        \int_{-\infty}^\infty \frac{y(u)}{(x - u)^2 + a^2} \id{u} &=
        \frac{1}{x^2 + b^2},
    \end{align*}
    onde $0 < a < b$, \'{e} dada por
    \begin{align*}
        y(x) &= \frac{a}{b \pi} \frac{(b - a}{x^2 + (b - a)^2}.
    \end{align*}
    \begin{solution}
        % TODO Escrever solu\c{c}\~{a}o.
    \end{solution}

    \question Com $\mathcal{F}_s$ e $\mathcal{F}_c$ denotando, respectivamente,
    as transformadas em seno e cosseno de Fourier, mostre que
    \begin{parts}
        \part $\mathcal{F}_s\left[ \exp(-x) \cos(x) \right] = \sqrt{2 / \pi} k^3
        / \left( k^4 + 4 \right)$,
        \begin{solution}
            % TODO Escrever solu\c{c}\~{a}o.
        \end{solution}

        \part $\mathcal{F}_s\left[ \left( H(x) - H(x - \pi) \right) \sin(x)
        \right] = \sqrt{2 / \pi} \sin(k \pi) / \left( 1 - k^2 \right)$,
        \begin{solution}
            % TODO Escrever solu\c{c}\~{a}o.
        \end{solution}

        \part $\mathcal{F}_c\left[ x \exp(- a x) \right] = \sqrt{2 / \pi} \left(
        a^2 - k^2 \right) / \left( a^2 + k^2 \right)^2$,
        \begin{solution}
            % TODO Escrever solu\c{c}\~{a}o.
        \end{solution}

        \part $\mathcal{F}_c\left[ \exp(-a x^2) \right] = \left( 2 a
        \right)^{-1/2} \exp\left( -k^2 / (4 a) \right)$.
        \begin{solution}
            % TODO Escrever solu\c{c}\~{a}o.
        \end{solution}
    \end{parts}

    \question Mostre que
    \begin{parts}
        \part $\mathcal{F}_s\left[ f'(x) \right] = -k \mathcal{F}_c\left[ f(x)
        \right]$,
        \begin{solution}
            % TODO Escrever solu\c{c}\~{a}o.
        \end{solution}

        \part $\mathcal{F}_c\left[ f'(x) \right] = - \sqrt{2 / \pi} f(0) + k
        \mathcal{F}_s\left[ f(x) \right]$.
        \begin{solution}
            % TODO Escrever solu\c{c}\~{a}o.
        \end{solution}
    \end{parts}

    \question Seja $\erfc(x) = \left( 2 / \sqrt{\pi} \right) \int_\pi^\infty
    \exp(-t^2) \id{t}$. Usando a rela\c{c}\~{a}o acima entre
    $\mathcal{F}_x\left[ |f'(x)| \right]$ e $\mathcal{F}_s\left[ f(x)
    \right]$ mais o resultado para $\mathcal{F}_c\left[ \exp\left( -a x^2
    \right) \right]$, mostre que
    \begin{align*}
        \mathcal{F}_s\left[ \erfc(\lambda x) \right] &= \sqrt{\frac{\pi}{2}}
        \frac{1 - \exp\left( -k^2 / \left( 4 \lambda^2 \right) \right)}{k}.
    \end{align*}
    \begin{solution}
        % TODO Escrever solu\c{c}\~{a}o.
    \end{solution}
    
    \question Seja a equa\c{c}\~{a}o
    \begin{align*}
        y''(x) - k^2 y(x) &= f(x),
    \end{align*}
    onde $x \geq 0$, com as condi\c{c}\~{o}es
    \begin{align*}
        y'(0) &= a, & \lim_{x \to \infty} y(x) &< \infty.
    \end{align*}
    Use o m\'{e}todo da transformada de Fourier em cossenos para mostrar que
    \begin{align*}
        y(x) &= \frac{-a}{k} \exp(-k x) - \frac{1}{2k} \int_0^\infty f(\xi)
        (\exp(-k |x - \xi|) + \exp(-k | x + \xi|) \id{\xi}.
    \end{align*}
    \begin{solution}
        % TODO Escrever solu\c{c}\~{a}o.
    \end{solution}

    \question Usando a transformada em seno de Fourier, mostre que a
    solu\c{c}\~{a}o da equa\c{c}\~{a}o do calor ,
    \begin{align*}
        \devp{u}{t} & D \devp{^2 u}{x^2},
    \end{align*}
    onde $t > 0$, $0 < x < \infty$, com as condi\c{c}\~{o}es que $u(x, t)$,
    $\partial u(x, t) / \partial x$, e $\partial^2 u(x, t) / \partial x^2$ se
    anulem para $x \to 0$ e $x \to \infty$ e a condi\c{c}\~{a}o inicial
    \begin{align*}
        u(x, 0) &= f(x)
    \end{align*}
    \'{e} dada por
    \begin{align*}
        u(x,t) &= \frac{2}{\pi} \int_0^\infty \int_0^\infty f(\xi) \exp(-D
        k^2 t) \sin(k \xi) \sin(k x) \id{\xi} \id{k}.
    \end{align*}
    \begin{solution}
        % TODO Escrever solu\c{c}\~{a}o.
    \end{solution}
\end{questions}
% \bibliographystyle{plain}
% \bibliography{bibliography}
\end{document}
